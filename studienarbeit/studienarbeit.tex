\documentclass[a4paper,
DIV=13,
12pt,
BCOR=10mm,
department=FakIM,
%lucida,
%KeepRoman,
oneside,
parskip=half,
automark,
listof=totocnumbered,
bibliography=totocnumbered,
acronym=totocnumbered
%headsepline,
] {OTHRartcl}

\usepackage[utf8]{inputenc}
\usepackage[a4paper, margin=0cm, left=3.5cm, top=2.5cm, right=2.5cm, bottom=3.0cm]{geometry}
\usepackage[ngerman]{babel}
\usepackage[printonlyused]{acronym}
\usepackage{graphicx}
\usepackage{hyperref}
\usepackage{amsmath}
\usepackage{minted}

\newcommand{\comment}[1]{}
\date{\today}
\title{Vulnerability eingebetteter Systeme}
\author{Gruber Daniel}
\studentid{3214109}
\documenttype{Studienarbeit}
\department{Informatik}
\startingdate{03.\,Juni 2022}
\closingdate{20.\,Juli 2022}

\firstadvisor{Schmidt Jonas}
%\secondadvisor{Herr Altmann}
%\externaladvisor{Dr. Klara Endlos}
%\externallogo[height=1.5cm]{firmenlogo}


%\ohead*{\date}
%\chead{}
%\ihead{\date}
%%\ohead{\date}
%\ofoot{}
%\cfoot{}
%\ifoot{Daniel Gruber}
%\ofoot*{\pagemark}% Seitenzahl in die Mitte des Fußes, auch auf plain Seiten

% Zeilenabstand auf 1.5
\usepackage[onehalfspacing]{setspace}
\newcommand*{\quelle}[1]{\par\raggedleft\footnotesize Quelle:~#1}

\begin{document}
\maketitle

\tableofcontents
\newpage

\section*{Abkürzungsverzeichnis}
\label{abkuerzungsverzeichnis}
\begin{acronym}[AUTOSAR]
  \acro{aslr}[ASLR]{Address Space Layout Randomization}
  \acro{dep}[DEP]{Data Execution Prevention}
  \acro{ecu}[ECU]{Electronic Control Unit}
  \acro{lr}[LR]{Linking Register}
  \acro{mmu}[MMU]{Memory Management Unit}
  \acro{mpu}[MPU]{Memory Protection Unit}
  \acro{pc}[PC]{Program Counter}
  \acro{risc}[RISC]{Reduced Instruction Set Computer}
  \acro{rop}[ROP]{Return Orientated Programming}
\end{acronym}
\newpage

% ============================================================================== Kapitel 1: Einleitung ===========================================================================================================
\section{Einleitung}
\label{sec:Einleitung}
Embedded Systems befinden sich in vielen Gegenständen und Geräten unseres täglichen Lebens, auch beispielsweise im Fahrzeug.
Die Vernetzung im Fahrzeug und damit auch die Anzahl von Mikrocontrollern nehmen im aktuellen Jahrzehnt deutlich zu, weshalb in Deutschland unter anderem die Automobilhersteller im Bezug auf Sicherheit des Fahrzeugs im Fokus stehen.
Autombilhersteller wie BMW, AUDI und Daimler sind erst seit einigen Jahren vertieft im Bereich der Softwareentwicklung tätig, worunter insbesondere die Entwicklung eines autonom fahrenden Fahrzeugs und die Entwicklung
bzw. mittlerweile Erweiterung an Funktionalität des Infotainmentsystems zählen. Nicht nur, weil diese Automobilhersteller relativ neu in der Softwareentwicklung im Vergleich zu den Technologieriesen wie Google, Facebook und Co.
sind, sondern insbesondere wegen der wenigen Schutzmechanismen in Embedded Systems, treten hier bereits längst bekannte Schwachstellen verlgeichsmäßig oft auf.
Darunter fallen Schwachstellen wie die memcmp Timing Attacke als Beispiel für einen Seitenkanalangriff, Buffer Overflows und Format String Vulnerabilities. \cite{car format string vuln.}

In dieser Studienarbeit werden diese drei genannten Schwachstellen detailliert beschrieben, wobei vorneweg konkret auf die STM32 Architektur eingegangen wird.
Zusätzlich werden zu jeder der dargestellten Schwachstellen deren mögliche Präventions- und Schutzmaßnahme vorgestellt.
Abschließend wird die Skalierbarkeit eines Angriffes basierend auf der Format String Vulnerabilty auf das Infotainmentsystem eines Fahrzeuges aufgegriffen und besprochen.


% ============================================================================== Kapitel 2: Vorstellung wichtiger Rahmenbedinungen ===============================================================================
\section{Vorstellung wichtiger Rahmenbedingungen}
\label{sec:VorstellungwichtigerRahmenbedinungen}
% Architekturmerkmale, Adressierungsmoid, Ausführungsmodelle, Betriebssstem
% Architektur des STM32F103C8T6 ARM®Cortex®-M3 32-bit RISC core, 72MHz, 64KB (64Kx8) Flash
% ARM v5 32 little endian
% Adapted Harvard Architektur: getrennte Busse für Befehle und Daten, aber beides liegt im Flash
%STM32F103C8T6	32Bit MCU, ARM CORTEX M3, Flash 64KB, RAM 20KB, IO 37, A/D 10x12b, 72MHz, LQFP48
%Die Cortex-M3-Architektur kann somit als Nachfolger für den ARM7 betrachtet werden und stellt mehr Rechenleistung als ein ARM7 bei geringerer Komplexität des Programmiermodells sowie geringerer Chipfläche zur Verfügung. Andere Untergruppen, wie z. B. der M1, sind z. B. für die Implementierung auf einem FPGA verfügbar.

Die STM32 Mikrocontroller-Familie wird vom europäischen Halbleiterhersteller STMicroelectronics N.V. produziert, welche als eine der ersten Hersteller unter anderem
die CORTEX M3 Lizenz von der Firma ARM erworben hat.
Die STM32 Controllerfamilie zeichnet sich durch einen 32-Bit ARM Cortex-M0/M3/M4 CPU aus, die speziell für Mikrocontroller neu entwickelt wurde.
ARM ist ein \ac{risc}, welche den Vorteil von insbesondere einen kompakten Befehlssatz sowie vielen Registern hat.
% TODO: RISC Genauer erklären

% BUFFER OVERLOW: wichtig, dass LR register da ist, da wird stack pointer gesichert und geladen, deswegen 2 funktionen um quasi return adr. zu überschreiben,
% - Stores return address in LR
% - Returning implemented by restoring the PC from LR
% - For non-leaf functions, LR will have to be stacked
% TODO: überdenken, ob HARVARD architektur noch genauer beschreiben
Der Cortex M0/M3/M4 Prozessor, insbesondere der Cortex M3 Prozessor wie beim STM32F103C8T6 vorhanden, basiert grundlegend auf der Harvard Architektur.
Hierbei existieren, wie für die Harvard Architektur typisch, verschiedene Busse für Befehle und Daten, welche es ermöglichen, zugleich Befehle und Daten zu lesen bzw. Daten in den Speicher zurückzuschreiben.
Aus Programmierersicht ist die CPU aber ein Von-Neumann Modell, da zwar die Trennung zwischen Befehls und Datenbus existiert, jedoch sowohl Befehle und Daten im gleichen Speichermedium, dem Flash, liegen und
somit der Adressraum dementsprechend linear programmiert werden kann.
Grundsätzlich wird dies in der Literatur als Modifizierte Harvard Architektur bezeichnet, da es zwar verschiedene Busse für Daten und Befehle gibt, jedoch keine strikte Trennung zwischen Daten und Befehlsadressraum gegeben ist.
Insbesondere ist kein getrennter physikalischer Speicher für Daten und Befehle vorhanden, denn beides befindet sich im Flash, worauf sowohl der Datenbus (DBUS) und Befehlsbus (IBUS) zugreift, wie auf
nachfolgender Abbildung zu sehen.
\begin{figure}[ht!]
  \begin{center}
    \includegraphics[scale=0.58]{figures/stm_architecture.png}
    \caption{STM Architecture}
    \label{fig: STM Architektur}
  \end{center}
\end{figure}
Dabei sichert man sich den Vorteil der Harvard-Architektur, dass gleichzetiges Laden von Befehlen und Daten für bessere Performance möglich ist, jedoch
verliert man die Eigenschaft des pyhsikalisch getrennten Adress- bzw. Speicherbereichs. Hier wird wie in der Von-Neumann Archiktur Daten und Befehle auf dem gleichen Speichermedium abgespeichert, welche durch verschiedene Adressbereiche getrennt sind. Da dies aber keine getrennten physikalischen Speicherbereiche sind, können beispielsweise Schwachstellen wie Buffer Overflows und Return Orientated Programming ausgenutzt werden, um mit eigenen Daten den Programmfluss bzw. Befehle zu modifizieren.
\cite{ARM Modell Overview} \cite{STM32 Mikro Prozessorarchitekturen} \cite{STM32 Mikro STM32} \cite{STM32 Controller Lecture}

Manche Architekturen bieten des Weiteren an, die Ausführung von Daten zu verhindern, indem eine Hardware unterstützte \ac{mmu} oder \ac{mpu} entsprechend konfiguriert wird.
Die \ac{mmu} besitzt in der Regel mehr Funktionalitäten als die \ac{mpu}, welche sich nur auf Speicherschutz konzentriert.
Da die STM32 Mikrocontroller Familie wenn dann nur eine \ac{mpu} besitzt, wie z.B. der STM32F303, wird hier nur darauf eingegangen.
Die \ac{mpu} ermöglicht es, ein eingebettetes System robuster und sicher zu machen, indem beispielsweise der SRAM bzw. Bereiche vom SRAM als nicht-ausführbar definiert werden können
und damit bestimmte Schwachstellen nicht mehr ausgenutzt werden können. \cite{MPU}
%Des Weiteren besitzen die meisten STM32, insbesondere der in der Übung verwendete STM32F103C8T6, eine \ac{mpu}.
% TODO: MPU ausführlich erklären hier!!
% verbieten, dass User applikation von korrupten dataen verwendet von kritischen tasks wie os kernel
% Memmory access attributes ändern

%%%%%% SPEICHERMODELL
Das Speichermodell bzw. der Adressierungsbereich von möglichen 4GB der CPU ist in \autoref{fig: Memory map} dargestellt.
Teil dieses Adressierungsbereichs sind der Code, der sich im Flash befindet, und der SRAM, welche beide in \autoref{fig: Flash und SRAM} dargestellt sind.
Dabei ist insbesondere wichtig, dass der Stack nach unten, d.h. von höheren zu niedrigen Adressen wächst, was es ermöglicht,
die Rücksprungadresse und andere Bereiche im SRAM über einen Buffer Overflow oder eine Format String Vulnerabilty für Angriffe auszunutzen. \cite{STM32F103Cx DataSheets}
\begin{figure}[ht!]
  \begin{minipage}[b]{.45\linewidth}
    \includegraphics[scale=0.47]{figures/memory_model.png}
    \caption{Memory map}
    \label{fig: Memory map}
  \end{minipage}
  \hspace{.1\linewidth}
  \begin{minipage}[b]{.48\linewidth}
    \includegraphics[width=\linewidth]{figures/flash_sram.png}
    \caption{Flash und SRAM}
    \label{fig: Flash und SRAM}
  \end{minipage}
\end{figure}

% Stack Pointer, PC Register, LR Register
Auch die Funktionsweise des \ac{lr} Register in Kombination mit dem \ac{pc} Register ist für die konkrete Ausnutzung nachfolgender Schwachstellen bedeutend.
Denn das \ac{lr} speichert die Rücksprungadresse der Funktion, welche nachfolgend dann in den \ac{pc} geladen wird. Der \ac{pc} gibt an, wo das Programm fortgesetzt wird.
Erst wenn mehrere Funktionen vorhanden sind, wird die Rücksprungadresse auf dem Stack gespeichert und muss von
diesem wieder in das \ac{lr} geladen werden, welches dann wiederum in den \ac{pc} geladen wird. \cite{ARM Modell Overview}
% TODOO: SCHREIBEN WIESO QUASI 2 FUNKTIONEN BENÖTIGT WERDEN
Deswegen werden für das Überschreiben der Rücksprungadresse zwei Funktionen benötigt, wobei eine die andere aufruft, um das Überschreiben der Rücksprungadresse der äußeren Funktion zu ermöglichen.

\autoref{fig: ARM Register Set} zeigt die vorhandenen Register im STM32F103C8T6 auf, wobei neben den genannten Registern noch der Stack Pointer und die General Purpose Register zu erwähnen sind. \cite{STM32F103Cx DataSheets}
\begin{figure}[ht!]
\begin{center}
  \includegraphics[scale=0.5]{figures/arm_register_set_selfmade.png}
  \quelle{\cite{STM32F103Cx DataSheets}}
  \caption{ARM Register Set}
  \label{fig: ARM Register Set}
\end{center}
\end{figure}

%%%% LITTLE ENDIAN
Außerdem ist hier zu erwähnen, dass die STM32 Mikrocontroller-Familie standardmäßig auf Little Endian setzt, d.h. das niederwertigste Byte befindet sich an der niedrigsten Adresse. %\cite irgendein manual
Die Abspeicherung in Little Endian spielt insbesondere für die Schwachstelle Buffer Overflow eine wichtige Rolle, da beim Auslesen des
Speichers dies zu berücksichtigen ist. Aber auch bei der Format String Vulnerability muss beim Überschreiben der Adresse auf die Endianness geachtet werden. \cite{STM32F103Cx DataSheets} \cite{STM32F10xxx/20xxx/21xxx/L1xxx3Cx8 DataSheets}

Wie viele Embedded Systems haben die Mikrocontroller der STM32 Familie kein Betriebssstem, das weitere Schutzmechanismen oder ähnliche Dienste bieten würde.

% ============================================================================== Kapitel 3: Schwachstellen =======================================================================================================
\section{Schwachstellen}
\label{sec:Schwachstellen}
%%%%% MEMCMP TIMING ATTACKE
\subsection{memcmp Timing Attacke für Bruteforcing}
\subsubsection{Beschreibung}
Die memcmp Timing Attacke ist ein typischer Seitenkanal-Angriff. Dieser Angriffstyp basiert auf Informationen, die von der konkreten Implementierung eines Systems abhängen.
Bei der memcmp Timing Attacke basiert dies auf dem Wissen über die benötigte Zeit eines Vergleichs von Speicherbereichen, welcher in der Software implementiert ist. \cite{Hardware Hacking Handbook}
Vorausgesetzt eine Speichervegleichsfunktion ist so implementiert, dass beim ersten nicht übereinstimmendem verglichenen Zeichen von der Funktion 'false' zurückgegeben wird,
benötigt der Vergleich unterschiedlich lange, je nach Anzahl richtiger Zeichen einer Zeichenkette.
Hier wird konkret der Aspekt der Zeit ausgenutzt, denn die Dauer der Funktion hängt von der Anzahl und Richtigkeit der zu vergleichenden Speicherbereichen bzw. Zeichenketten ab.
Je länger die Funktion benötigt, desto mehr Buchstaben waren beim entsprechenden Vergleich richtig.
Diese Information der Dauer einer Funktion je nach Vergleich kann man nun ausnutzen, um Bruteforcing bei Passworteingaben deutlich zu optimieren.
Bei reinem Bruteforcing müssen alle Kombinationen betrachtet werden, d.h.
bei einem Passwort der Länge 6 sind $|A|*|A|*|A|*|A|*|A|*|A| = |A|^6$ Kombinationen möglich, wobei $|A|$ die Mächtigkeit der möglichen Eingabezeichen ist.
Dahingegen kann bei einer memcmp Timing Attacke Stelle für Stelle durchprobiert werden, und die Auswahl für die jeweilige Stelle, die am längsten benötigt hat,
wird als richtig übernommen, denn dann hat die Funktion für die jeweilige Stelle einen erfolgreichen Vergleich durchgeführt.
Dies führt dazu, dass die nächste Stelle überprüft wird, was insofern bedeutet, dass die Funktion mehr Iterationen zu durchlaufen hat und damit mehr Zeit benötigt.
Insgesamt führt die memcmp Timing Attacke also zu einer erheblichen Verbesserung, indem beim Fall der Passwortlänge von 6
nur bis zu $ |A|+|A|+|A|+|A|+|A|+|A| = 6 * |A| $ Kombinationen auszutesten sind. \cite{Hardware Hacking Handbook}

\textit{Anmerkung} \mbox{} \\
In der Realität liegt solch ein Vergleich im Bereich von Nanosekunden, da nur wenige CLock Cycles für den Vergleich benötigt werden.
Dies bedeutet, dass der Delay über ein USB Kabel deutlich größer ist (im Millesekunden Bereich) als die Dauer des Vergleichs.
Aus diesem Grund werden für solche memcmp Timing Attacken Oszilloskope oder Logic Analyzers benötigt, um den Zeitunterschied
für den Vergleich am Embedded System zu messen. \cite{Hardware Hacking Handbook}

\subsubsection{Beispiel}
In diesem Abschnitt wird ein repräsentatives Beispiel für oben beschriebene Schwachstelle dargestellt.
Der Einfachheit halber wird ein PIN Vergleich der begrenzten Länge 4 durchgeführt, wobei das Alphabet 0-9 ist, d.h. eine Mächtigkeit von $|A| = 10$ besitzt.
Dies gilt ohne Einschränkung der Allgemeinheit und kann beliebig in der Länge sowie der Mächtigkeit des Alphabets verändert werden.
Zudem wird die Annahme getroffen, dass der PIN Vergleich erst nach vollständiger PIN Eingabe erfolgt. Dabei
wird folgende Methode für die Überprüfung des PINs verwendet:
\begin{figure}[ht!]
  \begin{center}
    \includegraphics[scale=0.8]{figures/CodeSnippets/CodeSnippet_memcmp_pin_correct_base.png}
    \caption{memcmp Timing Attacke - Methode zur Überprüfung eines Pins}
    \label{fig:memcmpcheckpinbase}
  \end{center}
\end{figure}

Für reines Raten, d.h. Bruteforcing ohne weitere Kenntnisse, sind $ 10*10*10*10 = 10^4 $ Kombinationen auszutesten.
Um die Anzahl der Kombinationen deutlich zu reduzieren, kann man den Vorteil des Wissens über die Implementierugn der oben dargestellten Funktion nutzen und damit das Prinzip der memcmp Timing Attacke verwenden.
Denn die Methode für das Überpüfen des PINs gibt beim ersten nicht korrekten Zeichen 'false' zurück, weshalb die die Ausführungsdauer der
Funktion von der Anzahl der richtig eingegebenen PIN Stellen abhängt.
Dafür wird Stelle für Stelle durchgegangen und überprüft, angefangen bei der ersten Stelle für jede mögliche Eingabe von 0-9, welche Eingabe die längste Zeit benötigt hat.
Denn wenn die Stelle richtig ist, war der Vergleich richtig und die Funktion wird die nächste Stelle überprüfen, was mit einer längeren Dauer für die Funktion
verbunden ist.
Konkret für die erste Stelle werden also alle Möglichkeiten durchgetestet von \textit{0000, 1000, 2000 bis 9000}, wobei für jeder dieser Eingaben
eine Zeitmessung durchgeführt wird.
Für die erste Eingabestelle stellt \autoref{fig: Timing Attack} oben dar, wie sich die Vergleichszeit im
korrekten Fall ($t_correct$) zum Fehlerfall ($t_bad$) unterscheidet.
Da der korrekte Pin 1337 ist, wird für die Eingabe 1000 die Vergleichszeit länger dauern, wie in $t_correct$ dargestellt.
Für alle anderen Möglichkeiten wird das obere Diagramm mit $t_bad$ zutreffen.
\begin{figure}[ht!]
  \begin{center}
    \includegraphics[scale=0.28]{figures/timing_attack_duration_example_cropped.jpeg}
    \quelle{\cite{Hardware Hacking Handbook}}
    \caption{Timing Attack}
    \label{fig: Timing Attack}
  \end{center}
\end{figure}
Diese Angriff wird für jede Möglichkeit der nächsten Stelle bis zur letzten Stelle druchgeführt ausgehend vor der korrekten Eingabe der jeweils vorherigen Stellen.
Bei der letzten Stelle ist die Zeitmessung überflüssig, denn im korrekten Fall hat man das System entsperrt.
Der Vorteil dieser Methode ist, dass die PIN Stellen sequentiell ausgehend vom Wissen über die Position richtig erraten werden.
Damit erreicht man, dass die maximale Anzahl an Kombinatione maximal $10+10+10+10 = 4*10 = 40 $ beträgt.
Das bedeutet, dass die Möglichkeiten bei der memcmp Timing Attack für Bruteforce gegenüber reinem Bruteforce nur noch $\frac{40}{1000} = \frac{4}{100} = 4\% $
aller Möglichkeiten betragen. \cite{Hardware Hacking Handbook}

\subsubsection{Prävention/Schutzmaßnahmen}
Für obige Funktion gibt es eine Vielzahl von Schutzmaßnahmen, die im Wesentlichen solche Angriffe deutlich erschweren, aber nicht 100\%ig verhindern.
Bei der Annahme, dass das Passwort in Klartext überprüft wird und nicht als gehashter Wert, werden insgesamt 4 Schutzmaßnahmen vorgestellt.

Die erste Schutzmaßnahme zielt auf eine korrelationslose bzw. konstante Zeit bei der Überprüfung ab. Dies wird erreicht, indem unabhängig
von einer falschen Stelle immer alle Stellen überprüft werden und erst dann das Ergebnis des Vergleichs zurückgegeben wird.
Hierbei wäre für oben dargestellten Code genau eine Änderung nötig, nämlich die Verwendung einer boolschen Variable,
die defaultmäßig true ist und bei einem fehlerhaften Vergleich auf false gesetzt wird. Dabei ist zu beachten, dass
alle Stellen überprüft werden und erst am Ende das Ergebnis des Vergleichs zurückgegeben wird. Die dafür notwendige Änderung ist in \autoref{fig:memcmpcheckpinimproved}, insbesondere in Zeile 3, dargestellt.
\begin{figure}[ht!]
  \begin{center}
    \includegraphics[scale=0.8]{figures/CodeSnippets/CodeSnippet_memcmp_pin_correct_improved_1.png}
    \caption{memcmp Timing Attacke - verbesserte Methode zur Überprüfung eines Pins}
    \label{fig:memcmpcheckpinimproved}
  \end{center}
\end{figure}
Eine ähnliche Schutzmaßnahmen, die zwar nicht auf konstante Zeit setzt, sondern auf Randomisierung von Zeit, kann durch Hinzufügen von
zufällig ausgeführten Operationen jeglicher Art, z.B. \textit{sleep()} Statements, implementiert werden. Dies erschwert die Korrelation von gemessener Zeit und korrekter bzw. fehlerhafter Eingabe.

Auch basierend auf der Randomisierung könnte die Erweiterung helfen, den Start des Vergleichs zu randomisieren, d.h. bei jeder neuen Eingabe wird der Vergleich bei
einer anderen Position startend durchgeführt.

Eine weitere Schutzmaßnahme ist die Ausführung von \textit{Decoy Operations} (dt. Ablenkungsmanöver), wobei Algorithmen zufällige Berechnungen durchführen.
Dabei werden beispielsweise manche Werte zufällig öfter verglichen oder die Ergebnisse von zufälligen weiteren Berechnungen nicht verwendet. \cite{Hardware Hacking Handbook Chapter 14}

Alle vorgestellten Maßnahmen zielen darauf ab, Seitenkanalinformationen zu verfälschen bzw. damit die Nachvollziehbarkeit des Algorithmus zu erschweren.

%%%%% FORMAT STRING VULNERABILITY
\subsection{Format String Vulnerabilty}
\subsubsection{Beschreibung}
Eine Format String Vulnerabilty tritt auf, wenn eine Benutzereingabe als Befehl interpretiert wird.
Weitergeführt kann ein Angreifer dies ausnutzen, um Code auszuführen, den Stack auszulesen oder gezielt das Programm durch einen Segmentation Fault zum Absturz bringen.
Diese Schwachstelle basiert auf variadische Funktionen, d.h. Funktionen, die eine variable Anzahl an Argumenten akzeptieren.
Eine von jedem C Programmierer benutzte variadische Funktion ist beipsielsweise die \textit{printf} Funktion. \cite{OWASP Format String Vuln.}
Das erste Argument einer printf Funktion ist der sogenannte Format String, der mit den angegebenen Paramtern beginnend mit \%, wie \%s, \%d, \%x usw., besimmt, wie nachfolgende Parameter
als Argumente verwendet werden.
\autoref{fig: printf - Format String} verdeutlicht dies, wobei \textit{name} ein string und \textit{age} eine integer Variable ist, die in den entsprechenden
Paramtern \%s und \%d als Argumente ersetzt werden.
% Parameter im Format String beginnen mit %
\begin{figure}[ht!]
  \begin{center}
    \includegraphics[scale=0.5]{figures/format_string_vulnerability_format_string.png}
    \caption{printf - Format String}
    \label{fig: printf - Format String}
  \end{center}
\end{figure}

Falls die printf - Funktion unsicher programmiert ist, wie in \autoref{fig:formatstringreadbase} zu sehen,
wird diese Funktion ohne explizite Parameter aufgerufen, weshalb die Werte von den Registern bzw. weiterführend vom Stack ausgelesen werden.
Allgemein funktiert dies, wenn der Format String nicht der Anzahl der enstprechenden nachfolgenden Argumente entspricht, weshalb dann unsichere Speicheroperationen, wie Zugriff auf Register bzw. Stack,
durchgeführt werden.
\begin{figure}[ht!]
  \begin{center}
    \includegraphics[scale=0.7]{figures/CodeSnippets/CodeSnippet_formatstring_base.png}
    \caption{Format String Read Vulnerability - Beispiel}
    \label{fig:formatstringreadbase}
  \end{center}
\end{figure}
Durch das Auslesen von Registern, Stack und Speicher kann ein Angreifer wertvolle Informationen über das laufende Progamm gewinnen.
Dazu können beipsielsweise Passwörter zählen, die im Arbeitsspeicher (RAM) oder in anderen Speicherbereichen vorzufinden sind.

Des Weiteren ist es möglich, wie anfangs erwähnt, eigenen eingegebenen Code auszuführen. Dabei ist oft das Ziel,
ein Shell zu öffnen, auf der weitere Aktionen ausgeführt werden können.
Dabei wird erst der Shell Start in Assembly gesucht. Mit der Format String Vulnerabilty wird darauffolgend mit \%x die Rücksprungadresse
der jeweiligen \textit{print} Funktion herausfinden. Nachfolgend kann die Schwachstelle ausgenutzt werden, indem im Input die Adresse der Rücksprungadresse von printf geschrieben wird
und daraufhin \%n ausgenutzt wird, um an diese Adresse die Addresse zum Ausführen der Shell zu schreiben. \cite{Format String Exploits}
% Anmkerung: Für C Programmierer ist derFehler offensichtlich und wird i.d.R. durch richtiges Programmieren kommplett ausgehebelt,
% jedoch kann es sein dass dieser printf vulnerability noch in anderen aufgerufenen Funktione wie syslog oder irgnedwelchen libs enthalten ist,
% deswegen ist der Fehler nicht so trivial wie er auf dem ersten Blick erscheitn.

\subsubsection{Beispiel}
Folgendes Beispiel konzentriert sich auf die beispielhafte Anwendung der Format String Vulnerabilty in Bezug auf das Auslesen von Speicher.
%Konkret soll hier der RAM ausgelesen werden, in welchen sich das Passwort nach erstmaligen Vergleich geladen wurde.
Der Codeausschnitt in \autoref{fig:formatstringwriteextended} zeigt ein Programm, dass eine printf Implementierung mit oben beschriebener Schwachstelle aufweist.
\begin{figure}[ht!]
  \begin{center}
    \includegraphics[scale=0.8]{figures/CodeSnippets/CodeSnippet_formatstring_extended.png}
    \caption{Format String Write Vulnerability - Beispiel}
    \label{fig:formatstringwriteextended}
  \end{center}
\end{figure}
Hierbei wird der Nutzer nach zwei Eingaben gefragt, nämlich dem Alter und dem gewünschten Benutzernamen.
Letztere Eingabe weist die entsprechende Format String Schwachstelle auf.
Hierbei kann der Nutzer bzw. der Angreifer dies ausnutzen, indem er als Alter eine beliebige Zahl eingibt, die als Adresse für die auslzusende Speicherzelle dient.
In der zweiten Eingabe, der Benutzername eingabe, kann man nun mit enstprechend vielen \textit{\%d} und dann einem \textit{\%s}, welches sich genau an der Position des
vorher gegebenen Alters eingibt. Beispielsweise kann die Eingabe dann wie folgt aussehen: \textit{\%d\%d\%s}.
Das eingegebene Alter bzw. die Adresse der Speicherstelle, die ausgelesen werden soll, wird mit \textit{\%s} als Pointer interpretiert und ausgegeben.
Hiermit kann man beispielsweise nun den RAM, den Flash oder andere Speicherbereiche auslesen.

\subsubsection{Prävention/Schutzmaßnahmen}
Für die Format String Vulnerabilty existiert eine einfache sehr effektive Schutzmaßnahme, nämlich
das sichere Programmieren, indem man die Paramter \textit{\%s}, \textit{\%d} oder weitere korrekt benutzt.
Dabei wird man zusätzlich von sicheren \textit{print} Funktionen unterstützt, wie z.B. \textit{sprintf}.
Weitere Schutzmaßnahmen in der Software könnten zudem noch sein, die Eingabe des Nutzers zu überprüfen, und
derartige möglicherweise schadhafte Eingaben nicht zuzulassen.

Des Weiteren gibt es Schutmaßnahmen, die in den Compilern implementiert sind, sodass die Compiler mindestens eine Warnung oder sogar einen Fehler
bei der Programmierung von unsicheren Code insbesondere in Zusammenhang mit \textit{print} Methoden anzeigen. \cite{OWASP Format String Vuln.}

%%%%% BUFFER OVERFLOW (ROP)
\subsection{Buffer Overflow und ROP}
\subsubsection{Beschreibung}
Ein Buffer Overflow tritt dann auf, wenn ein Programm mehr Daten in einen Buffer, beispielsweise ein Array in der Programmiersprache C, versucht zu speichern, als
dieser umfasst. \cite{OWASP Buffer Overflow}
Die Eingabe überschreitet die Grenzen des Buffers und überschreibt damit andere Daten, die außerhalb des Buffers gespeichert sind.
Diese Schwachstelle kann in zwei verschiedenen Kontexten auftreten, nämlich in einem Stack- oder in einem Heap Buffer Overflow.
In diesem Kapitel wird der Fokus auf den Stack Buffer Overflow gelegt, wobei man im Wesentlichen zwei verschiedenen Angriffsarten unterscheidet.
Ein Angriffstyp basiert auf der Idee eigenen Code einzufügen und diesen auszuführen, indem der Rücksprungadresse der Funktion so überschrieben wird, dass die Funktion
mit eigens eingefügten Code ausgeführt wird. Diese Art von Angriff ist relativ leicht zu verhindern, indem eine \ac{dep} verwendet wird. \cite{IEEE Xplore ROP}
Wie in Kapitel 2 erwähnt, ist hierfür die \ac{mpu} bei der STM Mikrocontroller Familie zuständig, der den SRAM und damit den sich dort befindenden Stack als nicht ausführbar markiert.

Die Verhinderung des zweiten Angriffstypes ist deutlich komplexer, denn hier werden vorhandene Funktionen zur Ausführung angebracht, indem die Rücksprungadresse mit der Adresse
dieser gewünschten Funktion überschrieben wird. %% TODO: hier noch schreiben, dass diese vorhandenen Funkt. nicht in SRAM, deswegen bringt DEP/MPU nix?
Dieser Angriffstyp ist als \ac{rop} Angriff bekannt, namensgebend durch die RETN Assembly Instruktion.
Im folgenden Beispiel wird ein \ac{rop} Angriff mit Berücksichtigung der Eigenschaften des STM32 dargestellt.

\subsubsection{Beispiel}
% sieh Beispiel in Übung, Skizze machen
Ein \ac{rop} Angriff funktioniert wie in \autoref{fig: Buffer Overflow} dargestellt.
Eine beispielhafte Funktion request\_input() lässt dem User über eine weitere Funktion, scanf(), eine Eingabe tätigen, wobei das zugrundeliegende Speicherobjekt beispielsweise ein C Array ist.
Damit der Angriff möglich ist, muss die Eingabe so gewählt werden, dass ein Buffer Overflow auftritt, d.h. mehr Daten in das Array gespeichert werden, als dieses umfasst.
O.B.d.A wird hier ein char Array der Größe 8 gewählt. Bei der Eingabe befindet man sich in der scanf Funktion, die nun beispielsweise eine Eingabe der Länge 24 entgegennimmt:
\textit{AAAA AAAA AAAA AAAA AAAA 0ROP}  (Leerzeichen nicht als Eingabe, nur zur Lesbarkeit)
Für die mögliche Eingabe wurden auf dem nach unten wachsenden Stack (d.h. von hohen zu niedrigen Adressen) nur 8 Byte reserviert.
Nun werden zu den höheren Adressen hin (der Stack wächst ja nach unten bei der Reservierung) weitere Daten überschrieben.
Bei der vorhandenen ARM Architektur, die sich im STM32 befindet, befindet sich wie eingangs erwähnt ein \ac{lr}, welches die Rücksprungadresse der Scanf Funktion noch hält, da keine weiter Funktion aufgerufen wird.
Deshalb funktioniert der Rücksprung von der scanf Funktion in die request\_input() Funktion, denn hierbei wird nur das \ac{lr} in den \ac{pc} geladen.
Dies ist auch der Grund, wieso zwei Funktionen dafür benötigt, werden, wobei die eine in der anderen genestet sein muss.

Bei der request\_input() Funktion ist auf dem Stack die Rücksprungadresse überschrieben worden, wobei das Überschriebene dann in das \ac{lr} geladen wird.
Dies wird daraufhin wieder in den \ac{pc} geladen, in dem Fall \textit{POR0}. Hier ist noch zu bemerken, dass das als Little Endian interpretiert wird, und somit
letztendlich zu der Funktion an der Adresse \textit{0x0ROP} gesprungen wird.
Der Rücksprung der aufgerufenen Funktion wird i.d.Regel fehlschlagen und das Programm abbrechen, jedoch bleiben die mit der gewünschten aufgerufenen Funktion bestehen.
Damit ist der ROP Angriff erfolgreich. \cite{IEEE Xplore ROP} \cite{OWASP Buffer Overflow Attack}
% Denn nun wird die Rücksprungadresse von request_input() mit der gewünschten ADresse der Funktion überschrieben
% Auch wichtig zu beachten ist Little Endian, was man bei den letzten 4 Byte besodners sieht, Eingabe ist 0ROP, wird aber im Stack little endian mäßgi gespeichert, d.h. POR0
\begin{figure}[ht!]
  \begin{center}
    \includegraphics[scale=0.56]{figures/BufferOverflow_w_stack.png}
    \caption{Buffer Overflow}
    \label{fig: Buffer Overflow}
  \end{center}
\end{figure}

\subsubsection{Prävention/Schutzmaßnahmen}
% Data execution Prevention (Memory protection in vorstellung wichtiger rahmbedigungen einfügen)
% Einige Microcontroller (e.g. STM32F303) besitzen eine Memory Protection
% Unit (MPU) mit speziellen Einstellunge
% 0. keine fehler machen
% 1. Data execution prevention (MPU=Memory Protection Unit)
% 2. Protection Rings
% 3. Address Space Layout Randomization (ASLR)
% 4. Stack Canaries
% 5. Watchdogs
Neben der offensichtlichen und schwierigsten Schutzmaßnahme, keine Fehler in der Softwareentwicklung zu machen, gibt es eine Vielzahl von Schutzmaßnahmen.
Eine davon, die \ac{dep} umgesetzt durch die \ac{mpu}, bereits in \autoref{sec:VorstellungwichtigerRahmenbedinungen} ausführlich erklärt, deklariert Speicherbereiche im SRAM bzw. Stack als nicht ausführbar
und verhindert damit den ersten Angriffstyp. Jedoch wird der zweite ANgriffstyp \ac{rop} nicht verhindert, da dieser auf bereits existierende Funktionen zugreift, die nicht im Stack liegen.

Für diesen Angriffstyp gibt es andere Schutzmaßnahmen wie die Nutzung von Protection Rings, \ac{aslr}, Stack Canaries oder Watchdogs.

Das Protection Rings Modell, welches von der Hardware unterstützt werden muss, dient der Trennung von Privilegien, mit denen verschiedene Prozesse laufen.
Die grundlegende Idee hinter diesem Modell ist, dass Prozesse nach dem Need-to-Know Prinzip nur mit den benötigten Rechten ausgeführt werden.
Dabei werden in diesem Modell die Ausführungsrechte bzw. -privilegien
als eine Menge von Ringen beschrieben, wobei diese hierarchisch von innen nach außen nummeriert werden.
Der innere Ring, Ring 0, bietet alle Rechte und Privilegien, während weiter außen liegende Ringe nur einen Teil der Privilegien des nächstinneren Ringes zur Verfügung stellen.
Üblicherweise wird dieses Modell so verwendet, dass das System mit den höchsten Privilegien gestartet und initalisiert wird, während Benutzer- und Applikationsprozesse mit niedrigerem Privilegien
laufen. Damit kann man erreichen, dass Applikationen mit Format String Schwachstelle nur noch soweit ausgenutzt werden, was die Privilegien, mit denen diese Applikation läuft, auszunutzen.
In der Regel wird dies soweit eingeschränkt, dass man einen Angriff verhindert oder zumindest dessen schadhafte Auswirkungen reduziert.

Eine weitere Möglichkeit \ac{rop} Angriffe zu verhindern, ist die Verwendung der \ac{aslr}, indem die Adressbereiche des Stacks, des Heaps und mögliche verlinkte Bibliotheken eines Programms unvorhersehbar und zufällig verteilt werden.
Da bei Angriffen es oft wichtig ist, die exakten Speicheradressen herauszufinden, ist diese Methode sehr effektiv, um die Ausnutzung von Schwachstellen durch gezieltes Springen zu einen Programm deutlich zu erschweren.
Eine detailliertere Umsetzung basiert beispielsweise darauf, nach definierten Zeitintervallen die Adressbereiche neu zu vergeben. \cite{OWASP Buffer Overflow} \cite{OWASP Buffer Overflow Attack}
%Eine weiterfMöglichkeit sind die Position-independen executables (PIE) . . .
%Damit kann ein Angreifer nicht gezielt zu Programmen springen, was die Ausnutzung von Schwachstellen erheblich erschwert.
% Fix #4.1: Position-independent executables (PIE)
% • Ein Programm kann beim kompilieren so gebaut werden, dass es nur relative
% Adressen zum PC benutzt (keine konstanten Adressen).
% • Das resultierende Executable wird PIE genannt und kann an einer zufälligen
% Adresse gespeichert und ausgeführt werden
% • In PIEs können statischer Code, Bibliotheken und Variablen zusätzlich durch
% zufällige Adressen geschützt werden um mehr Schutz gegen ROP zu bieten.

Des Weiteren gibt es eine Compiler basierte Lösung, nämlich unter anderem StackGuard, bei der ein Sicherheitsmechanismus im Compiler eingebaut ist, der korrupte bzw. modifizierte Rücksprungadressen überprüft und damit zumindest den Sprung dorthin verhindert.
Dabei wird ein ein "Stack Canary" - ein zufälliger bekannter Wert - im Stack neben der Rücksprungadresse geschrieben, während eine Kopie davon in einem Genera Purpose Register r0-r12 (siehe \autoref{fig: ARM Register Set})
gespeichert wird. Beim Abbauen der Funktion und damit implizit des Stacks wird beim Rücksprung der Stack Canary auf dem Stack mit der Kopie verglichen, um festzustellen, ob eine Buffer Overflow aufgetreten ist.
Eine ähnliche, sogar verbesserte Version, bietet der Stackshield unter Linux. Dieser bietet eine Speicherung der Rücksprungadresse an einem weiteren Platz, welche als letzte Operation vor Beenden der Funktion
dann wieder zurück geschrieben wird. \cite{IEEE Xplore ROP} 
% The application randomly generates the 32-bit or 64-bit canary values, so the application can detect improper modification of a canary value resulting from a buffer overflow with high probability. However, there exist attacks that can circumvent StackGuard's canaries to successfully corrupt return addresses and defeat the security of the system

Eine weitere Schutzmaßnahme sind Watchdogs, die wiederum in Software- und Hardware Watchdog unterteilt werden können. Diese funktionieren wie ein Timer Interrupt, welcher den Prozessor neustarten würde,
falls ein solcher auftritt. Bei Software Watchdogs werden vom Programm periodische Nachrichten gesendet, um den Timer zurückzusetzen, bei Hardware Watchdogs geschieht dies ähnlich, wobei dieser als physikalische Hardware
mit dem Prozessor verbunden ist. Sollten also in einem Programm Schwachstellen ausgenutzt werden, können die Watchdogs aufgrund nicht zurückgesetzten Timer, den Prozessor neustarten und damit die Ausnutzung vorerst verhindern. \cite{STM32 Watchdogs}

Zusammengefasst gibt es eine umfassende Menge von Schutzmaßnahmen, von denen hier einige dargestellt wurden. Anzumerken ist jedoch, dass diese Schutzmaßnahmen oft nur verhindern, dass schadhafter Code
ausgeführt wird, aber nicht, dass das System zum Absturz gebracht wird. Damit kann also die Verfügbarkeit des Systems beeinträchtigt werden, was in der englischen Literatur auch als Denial of Service (DOS) bekannt ist.
%%% MMU VS MPU überarbeiten, STM32f103 hat keine MPU oder MMU, STM32F303 z.B. schon


% ============================================================================== Kapitel 4: Besprechung der möglichen Skalierbarkeit ==============================================================================
\newpage
\section{Besprechung der möglichen Skalierbarkeit}
% https://cydrill.com/cyber-security/automotive-security-how-to-brick-your-car/
Die oben genannten Schwachstellen können auch in Infotainmentsystemen von Automobilen ausgenutzt werden mit dem Ziel, die Kontrolle über dieses zu erlangen.
Im Folgenden wird inbesondere die Ausnutzung der Format String Vulnerabilty im Infotainmentssystems eines Automobils betrachtet und des Weiteren auf mögliche Skalierungen eingegangen.

Mittlerweile entwickelt jeder größere Automobilhersteller sein eigenes Infotainmentsystem bzw. in Kooperation mit Softwareriesen wie Google, Apple, etc..
Die Funktionalität der Infotainmentsysteme steigt zunehmend an, von Nutzung eines Navigationssystem über die Verbindung mit Mobilgeräten und Nutzung von weiteren Apps.
Insbesondere Schnittstellen nach außen, wie die Bluetooth Kommunikation des Automobils mit einem Smartphone birgt einige Risiken.
Darunter zählt auch die Format String Vulnerability Schwachstelle. Diese konnte nämlich bei der Verbindung eines Mobilgerätes mit dem Infotainmentsystem über Bluetooth
ausgenutzt werden, indem der Name des mit sich zu verbindenden Smartphones beispielsweise auf \textit{\%c\%c\%c\%c\%c\%c\%c\%c} geändert wird.
Wie in der Format String Vulnerability erklärt, führt diese Eingabe bei unsicher programmierten \textit{printf} Statements dazu, dass Register bzw. der Stack ausgelesen werden.
Dies war in früheren Versionen des Infotainmentsystems möglich, und Daten, die von Experten als Karten- und Navigationinformationen identifiziert worden sind, konnten ausgelesen werden.

Allein das Auslesen von wichtigen Informationen des Automobils ist eine erhebliche Sicherheitslücke, jedoch kann dies weiter skaliert werden mit der Nutzung von \textit{\%s} oder \textit{\%n}.
Mit \textit{\%n} kann insbesondere \textit{nprintf()} in folgender Art und Weise ausgenutzt werden, dass ein Wert an die Speicheradresse geschrieben wird, die am Stack referenziert wird.
Der Angreifer kann also eine beliebige Adresse eingeben, dementsprechend \textit{\%n} ausnutzen, um dann an vorher eingegebene Adresse einen Wert zu schreiben.
Falls dies eine ausgesuchte Rücksprungadresse oder irgendeinen Pointer überschreibt, kann der Angreifer nun entweder eigens eingegebenen Code ausführen oder bereits bestehende Funktionen
ähnlich wie beim Buffer Overflow \ac{rop} Angriff ausführen.

Diese Schwachstelle kann also von einer Informationsgewinnung zur Ausführung von eigenem Code bei geschickter Anwendung von \textit{\%n} und dementsprechend vorliegenden Schwachstellen führen.
Bei vorliegender Schwachstelle kann der Anriff also so skaliert werden, dass anstatt nur Informationen ausgelesen werden, eigener Code ausgeführt wird (Remote Code Execution),
womit prinzipiell die Kontrolle über das Automibil erlangt werden kann.
Dies kann zu verschiedenen Angriffsszenarien führen. Ein Worst Case Szenario könnte sein, dass mit der Erlangung der Kontrolle über das Fahrzeug über Remote Code Execution angefangen beim
Informationssystem Lenk- und Bremsfunktionen manipuliert werden, die zu erheblichen negativen Auswirkungen führen können. \cite{car format string vuln.}

%some further data from the memory, like a password. But then it gets even worse: if the attacker uses the %n specifier,
%printf() will not READ, but instead WRITE a value (specifically, the number of characters printed out so far) to the memory address referenced on the stack.
%automotive security, secure coding in C and C++, format string
% If the attacker is crafty enough, they can exploit an arbitrary write vulnerability like this to overwrite a return address or any other code pointer,
% resulting in the CPU executing arbitrary code – so we have gone from a humble information leakage vulnerability to full Remote Code Execution (RCE)!
% RCE is the ‘holy grail’ of attackers; it is one of the most important concerns when it comes to secure coding, and especially secure coding in C and C++.
% Of course, these attacks are hard to carry out in practice, and thus they are quite popular in hacker contests (called CTFs).
% But hackers can be very persistent, especially if exploitation can allow them to – say – steal a high-value vehicle.

% ============================================================================== APPENDIX =========================================================================================================================
\begin{appendix}
\newpage
\listoffigures

\cleardoublepage
\newpage

%% BIBLIOGRAPHY
%% add citations with "\cite{source name}"
\begin{thebibliography}{99}
\bibitem{Hardware Hacking Handbook} Woudenberg, Jasper van and O'Flynn, Colin: \textit{The Hardware Hacking Handbook}, San Francisco, CA, USA: No Starch Press, 2022.
\bibitem{Operating System Concepts} Silberschatz, Abraham, Peter B. Galvin, and Greg Gagne: \textit{Operating System Concepts}, Wiley, 30. Jun. 2021.
\bibitem{Hardware Hacking Handbook Chapter 14} Woudenberg, Jasper van and O'Flynn, Colin: \textit{Chapter 14: Think of the Children: Countermeasures, Certifications and Goodbytes}, n.d.. Zuletzt aufgerufen am 09.07.2022. [Online]. https://github.com/HardwareHackingHandbook/notebooks/blob/main/labs/HHH\_14\_Think\_of\_the\_Children\_Countermeasures\_Certifications\_and\_Goodbytes.ipynb



\bibitem{car format string vuln.} Automotive Security, \textit{Automotive Security}. n.d.. Zuletzt aufgerufen am: 25.06.2022. [Online]. \\https://cydrill.com/cyber-security/automotive-security-how-to-brick-your-car/ % Für Besprechung der möglichen Skalierbarkeit automotive security (Einleitung und letztes Kapitel)
\bibitem{IEEE Xplore ROP} Prandini Marco und Ramilli Marco: \textit{Return-Orientated Programming}. 10. Dez, 2012. Zuletzt aufgerufen am: 09.07.2022. [Online]. https://ieeexplore.ieee.org/document/6375725
\bibitem{McKinsey}  Ondrej Burkacky: \textit{Automotive software and electronics 2030}. 2017. Zuletzt aufgerufen am: 09.07.2022. [Online]. https://www.mckinsey.com/~/media/mckinsey/industries/automotive\%20and\%20assembly/our\%20insights/mapping\%20the\%20automotive\%20software\%20and\%20electronics\%20landscape\%20through\%202030/automotive-software-and-electronics-2030-final.pdf
%Ondrej Burkacky, Munich Johannes Deichmann, Stuttgart Jan Paul Stein, Munich
%https://www.mckinsey.com/~/media/mckinsey/industries/automotive%20and%20assembly/our%20insights/mapping%20the%20automotive%20software%20and%20electronics%20landscape%20through%202030/automotive-software-and-electronics-2030-final.pdf
\bibitem{IEE buffer overflow stackguard} J.P. McGregor, D.K. Karig, R.B. Lee: \textit{A processor architecture defense against buffer overflow attacks}, 8. März 2004. Zuletzt aufgerufen am: 11.07.2022. [Online]. https://ieeexplore.ieee.org/document/1270612



% OWASP
\bibitem{OWASP Buffer Overflow}  OWASP: \textit{OWASP Buffer Overflow}. n.d.. Zuletzt aufgerufen am: 09.07.2022. [Online]. https://owasp.org/www-community/vulnerabilities/Buffer\_Overflow
\bibitem{OWASP Buffer Overflow Attack}  OWASP: \textit{OWASP Buffer Overflow Attack}. n.d.. Zuletzt aufgerufen am: 09.07.2022. [Online]. https://owasp.org/www-community/vulnerabilities/Buffer\_Overflow
\bibitem{OWASP Format String Vuln.}  OWASP: \textit{OWASP Format String Vuln.}. n.d.. Zuletzt aufgerufen am: 09.07.2022. [Online]. https://owasp.org/www-community/attacks/Format\_string\_attack

% DATASHEETS
\bibitem{MPU}  STM: \textit{Managing memory protection unit in STM32 MCUs}. n.d.. Zuletzt aufgerufen am: 09.07.2022. [Online]. https://www.st.com/resource/en/application\_note/dm00272912-managing-memory-protection-unit-in-stm32-mcus-stmicroelectronics.pdf
%https://www.st.com/resource/en/application_note/dm00272912-managing-memory-protection-unit-in-stm32-mcus-stmicroelectronics.pdf
\bibitem{STM32F103Cx DataSheets}  ST: \textit{STM32F103Cx8 u STM32F103xB DataSheets}. n.d.. Zuletzt aufgerufen am: 09.07.2022. [Online]. https://www.st.com/content/ccc/resource/technical/document/datasheet/33/d4/6f/1d/df/0b/4c/6d/CD00161566.pdf/files/CD00161566.pdf/jcr:content/translations/en.CD00161566.pdf
\bibitem{STM32F10xxx/20xxx/21xxx/L1xxx3Cx8 DataSheets}  ST: \textit{STM32F10xxx/20xxx/21xxx/L1xxx3Cx8 DataSheets}. n.d.. Zuletzt aufgerufen am: 09.07.2022. [Online]. https://www.st.com/resource/en/programming\_manual/pm0056-stm32f10xxx20xxx21xxxl1xxxx-cortexm3-programming-manual-stmicroelectronics.pdf

% CA QUELLEN
% IS Foliensatz: IS-03-Schwachstellen-2.pdf(für Buffer Overflow)
% CA folienSatz 01 ARM: file:///C:/Users/danie/00_Daten/06_OTHRegensburg_Semester6/04_Computerarchitektur/01_Vorlesungen/01_ARM.pdf ( für vorstellung wichtiger rahmenbed., RISC, HAVARD, VON NEUMANN, ...)
% CA FOliensatz 02 Speicherverwaltung: file:///C:/Users/danie/00_Daten/06_OTHRegensburg_Semester6/04_Computerarchitektur/01_Vorlesungen/02_Speicherverwaltung.pdf (für Format String VUln und Buffer Overflow)
% CA Foliensatz 03 Seitenkanäle: file:///C:/Users/danie/00_Daten/06_OTHRegensburg_Semester6/04_Computerarchitektur/01_Vorlesungen/03_Seitenkana%CC%88le.pdf (für memcmp timing attack)
\bibitem{ARM Modell Overview}  Bruce Hemingway: \textit{ARM Overview}. 2017. Zuletzt aufgerufen am: 09.07.2022. [Online]. https://courses.cs.washington.edu/courses/cse474/17wi/pdfs/lectures/03-arm\_overview.pdf (ARM)
\bibitem{Format String Exploits}  Code Arcana: \textit{Introduction to format string exploits}. 02. Mai 2013. Zuletzt aufgerufen am: 09.07.2022. [Online]. https://codearcana.com/posts/2013/05/02/introduction-to-format-string-exploits.html
% https://www.semanticscholar.org/paper/Architecture-Support-for-Defending-Against-Buffer-Xu-Kalbarczyk/743a19a7d832fc15a346329a0799d8988d4e7d8c?p2df

\bibitem{STM32 Mikro Prozessorarchitekturen} Mikrocontroller.net: \textit{Prozessorarchitekturen}. 25. Sep. 2018. Zuletzt aufgerufen am: 18.07.2022. [Online]. https://www.mikrocontroller.net/articles/Prozessorarchitekturen
\bibitem{STM32 Mikro STM32} Mikrocontroller.net: \textit{STM32}. 21. Jan. 2020. Zuletzt aufgerufen am: 18.07.2022. [Online]. https://www.mikrocontroller.net/articles/STM32

\bibitem{STM32 Controller Lecture} Kadah, Yassar M.: \textit{STM32 Microcontroller Lecture}. n.d.. Zuletzt aufgerufen am: 18.07.2022. [Online]. https://www.k-space.org/Class\_Info/STM32\_Lec2.pdf
\bibitem{STM32 Watchdogs} ST: \textit{STM32 Watchdogs}. Sep. 2021. Zuletzt aufgerufen am: 18.07.2022. [Online]. https://www.st.com/content/ccc/resource/training/technical/product\_training/group0/01/24/22/29/38/70/40/57/STM32WB-WDG\_TIMERS-Independent-Watchdog-IWDG/files/STM32WB-WDG\_TIMERS-Independent-Watchdog-IWDG.pdf/jcr:content/translations/en.STM32WB-WDG\_TIMERS-Independent-Watchdog-IWDG.pdf
\end{thebibliography}

\cleardoublepage
\makedeclaration
\end{appendix}

\end{document}
