\documentclass[a4paper,
DIV=13,
12pt,
BCOR=10mm,
department=FakIM,
%lucida,
%KeepRoman,
oneside,
parskip=half,
automark,
listof=totocnumbered,
bibliography=totocnumbered,
acronym=totocnumbered
%headsepline,
] {OTHRartcl}

\usepackage[utf8]{inputenc}
\usepackage[a4paper, margin=0cm, left=3.5cm, top=2.5cm, right=2.5cm, bottom=3.0cm]{geometry}
\usepackage[ngerman]{babel}
\usepackage[printonlyused]{acronym}
\usepackage{graphicx}
\usepackage{hyperref}
\usepackage{amsmath}

\date{\today}
\title{Vulnerabilty eingebetteter Systeme}
\author{Gruber Daniel}
\documenttype{Studienarbeit}
\department{Informatik}
\startingdate{14.\,März 2022}
\closingdate{23.\,Juli 2022}

\firstadvisor{Jonas Schmidt}
%\secondadvisor{Herr Altmann}
%\externaladvisor{Dr. Klara Endlos}

%\externallogo[height=1.5cm]{firmenlogo}


%\ohead*{\date}
%\chead{}
%\ihead{\date}
%%\ohead{\date}
%\ofoot{}
%\cfoot{}
%\ifoot{Daniel Gruber}
%\ofoot*{\pagemark}% Seitenzahl in die Mitte des Fußes, auch auf plain Seiten

% Zeilenabstand auf 1.5
%\usepackage[onehalfspacing]{setspace}

\begin{document}
\maketitle

\tableofcontents
\newpage

%Hier schonmal Abkürzungsberzeichnis vorbereitet. Im Text kann mit \ac{Abk} auf die Abkürzung referenziert werden. Es wird automatisch beim ersten referenzieren die volle Schreibweise genommen, ab dann immer die abkürzung
\section*{Abkürzungsverzeichnis}
\label{abkuerzungsverzeichnis}
\begin{acronym}[AUTOSAR]

\end{acronym}


% ============================================================================== Kapitel 1: Vorstellung wichtiger Rahmenbedinungen ===============================================================================
\section{Vorstellung wichtiger Rahmenbedinungen}





% ============================================================================== Kapitel 2: Schwachstellen ======================================================================================
\section{Schwachstellen}
\subsection{memcmp Timing Attacke für Bruteforcing}
\subsection{Format String Vulnerabilty}
\subsection{Buffer Overflow (ROP)}

% ============================================================================== Kapitel 3: Prävention ======================================================================================
\section{Prävention}


% ============================================================================== Kapitel 4: Besprechung der möglichen Skalierbarkeit ======================================================================================
\section{Besprechung der möglichen Skalierbarkeit}



% ============================================================================== APPENDIX ===============================================================================
\begin{appendix}
\listoffigures

\cleardoublepage
\begin{thebibliography}{99}
\bibitem{Nemo} Dr. Nemo: \textit{Submarines through the ages}, Atlantis, 1876.
\end{thebibliography}

\cleardoublepage
\makedeclaration
\end{appendix}

\end{document}
